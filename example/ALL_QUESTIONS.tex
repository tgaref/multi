%!TEX TS-program = xelatex
%!TEX encoding = UTF-8 Unicode
\documentclass[a4paper, 11pt]{article}
%

\usepackage[top=2.0cm,bottom=2cm,left=2.0cm,right=2.0cm]{geometry}
\usepackage{amsmath}
\usepackage{amssymb}
%
\usepackage[no-math]{fontspec}
\usepackage{xunicode}
\usepackage{xltxtra}
%\usepackage{xgreek}
\usepackage{graphicx}
%
\setmainfont{Linux Libertine O}

\newcommand{\fl}[1]{\lfloor {#1} \rfloor}
\newcommand{\ff}[1]{{\mathbb F}_{\! #1}}
\newcommand{\mat}{\mathrm{Mat}}
\newcommand{\Tr}{\mathrm{Tr}}
\newcommand{\F}{\mathbb F}
\newcommand{\J}{{\mathbb J}}
\newcommand{\D}{\mathbb D}
\newcommand{\N}{\mathbb N}
\newcommand{\Jac}{\mathbb J}
\newcommand{\R}{\mathbb R}
\newcommand{\Q}{\mathbb Q}
\newcommand{\C}{\mathbb C}
\newcommand{\Z}{{\mathbb Z}}
\newcommand{\de}{{\mathrm{det}}}
\newcommand{\Gal}{{\mathrm{Gal}}}
\newcommand{\cB}{{\mathcal B}}
\newcommand{\cN}{{\mathcal N}}
\newcommand{\cR}{{\mathcal R}}
\newcommand{\bb}{{\mathbf b}}
\newcommand{\bx}{{\mathbf x}}
\newcommand{\an}[1]{{\langle #1\rangle}}
\newcommand{\Aut}{\mathrm{Aut}}
\newcommand{\id}{\mathrm{id}}
\newcommand{\im}{\mathrm{im}}
\newcommand{\mkd}{\mathsf{\mu\kappa\delta}}
\newcommand{\ekp}{\mathsf{\varepsilon\kappa\pi}}
\title{  DIFFICULT }
\pagestyle{empty}
\begin{document}
\maketitle
\flushleft \underline{\bf Q2 }
\begin{enumerate}
\item Αυτή είναι η 1η εκδοχή της 2ης ερώτησης!
\begin{enumerate}
\item (true)  1η απάντηση $2^n$
\item (false)  2η απάντηση $\binom{2}{3}$
\item (false)  3η απάντηση $\frac{1}{2}$
\item (false)  4η απάντηση $2\cdot 3$
\end{enumerate}
\item Αυτή είναι η 2η εκδοχή της 2ης ερώτησης!
\begin{enumerate}
\item (true)  1η απάντηση $2^n$
\item (false)  2η απάντηση $\binom{2}{3}$
\item (false)  3η απάντηση $\frac{1}{2}$
\item (false)  4η απάντηση $2\cdot 3$
\end{enumerate}
\end{enumerate}
\flushleft \underline{\bf Q4 }
\begin{enumerate}
\item This is version 1 of question 4!
\begin{enumerate}
\item (true)  1 answer
\item (false)  2 answer
\item (false)  3 answer
\item (false)  4 answer
\end{enumerate}
\item This is version 2 of question 4!
\begin{enumerate}
\item (true)  1 answer
\item (false)  2 answer
\item (false)  3 answer
\item (false)  4 answer
\end{enumerate}
\end{enumerate}
\flushleft \underline{\bf Q3 }
\begin{enumerate}
\item This is version 1 of question 3!
\begin{enumerate}
\item (true)  1 answer
\item (false)  2 answer
\item (false)  3 answer
\item (false)  4 answer
\end{enumerate}
\item This is version 2 of question 3!
\begin{enumerate}
\item (true)  1 answer
\item (false)  2 answer
\item (false)  3 answer
\item (false)  4 answer
\end{enumerate}
\end{enumerate}
\flushleft \underline{\bf Q1 }
\begin{enumerate}
\item This is verision 1 of question 1!
\begin{enumerate}
\item (true)  1 answer
\item (false)  2 answer
\item (false)  3 answer
\item (false)  4 answer
\end{enumerate}
\item This is verision 2 of question 1!
\begin{enumerate}
\item (true)  1 answer
\item (false)  2 answer
\item (false)  3 answer
\item (false)  4 answer
\end{enumerate}
\end{enumerate}
\flushleft \underline{\bf Q5 }
\begin{enumerate}
\item This is version 1 of question 5!
\begin{enumerate}
\item (true)  1 answer
\item (false)  2 answer
\item (false)  3 answer
\item (false)  4 answer
\end{enumerate}
\item This is version 2 of question 5!
\begin{enumerate}
\item (true)  1 answer
\item (false)  2 answer
\item (false)  3 answer
\item (false)  4 answer
\end{enumerate}
\end{enumerate}
\end{document}
